\documentclass{article}
\usepackage{graphicx} % Required for inserting images
\usepackage{amsmath}
\usepackage{setspace} 
\usepackage{lineno}
\usepackage[margin=1in]{geometry}
\usepackage[
backend=biber,
style=apa,
sorting=nyt
]{biblatex}
\usepackage{float}
\usepackage{longtable}
\usepackage{caption}
\addbibresource{lamprey_refs.bib}

\title{Quantifying the cost of cultural bias in conservation decision-making}

\begin{document}

\date{}

\doublespacing

\linenumbers

\maketitle


\section{Abstract}

Human systems shape not only ecological data collection, but also perspectives and assumptions made during model building, generating uncertainty about how dominant value systems and cultural bias limit ecological inference. Understanding how this uncertainty propagates in conservation decision-making requires translation beyond abstract statistical measures and toward performance on real-world objectives. Here we develop a quantitative framework combining uncertainty quantification and decision theory to understand the value of reducing uncertainty generated by cultural bias for a conservation decision maker. We demonstrate this framework in the context of decision-making in a multi-species dynamical system in the Columbia River basin (CRB), where the Pacific lamprey – an ancient, foundational species in the CRB – modulates salmonid ocean survival both as a parasite of salmonids and a predation buffer against marine mammals. Despite its critical ecological role in the CRB and its high cultural value for Columbia basin Indigenous peoples, the Pacific lamprey has faced a precipitous decline in the last century, largely stemming from Euro-American cultural bias. Asymmetric measurement error and observation of local, rather than global, properties of the system limit understanding of density-dependence in species interactions, and subsequently evaluation of management actions. We quantify the cost of lamprey uncertainty in units relevant for a decision maker, demonstrating that the value of information around overlooked species can be considerable. These results highlight how Situated Modeling in ecological management can be used to interrogate how modeling as a process and practice is contextualized.

\section{Introduction}

Building an ecological model is an active exercise of bringing about the world \parencite{schluter2025disentangling}. Yet this world-making process is limited, as models tend to exist where the light is shining (Figure 1), often reflecting what is known and where the data is, rather than what is unknown. The presence or absence of parameters and processes in a model can reflect dominant systems of knowledge production, dominant societal values, human perceptions of species charisma and value, and cultural bias \parencite{silver2022fish, stoudt2022identifying, hughes2021sampling}. For decision-makers, uncertainties that are known and articulated already challenge management of ecological systems \parencite{polasky2011decision}, yet understanding how cultural bias and systems of power generate uncertainties that propagate in conservation decisions remain largely unexplored.

\begin{figure}[H]
    \centering    \includegraphics[width=1\textwidth]{Figure1_conceptualfigure-01.png}
    \caption{What is the cost of looking only under a lamp-post for a scientific solution, simply because that's where the light is? Here we present a methodological framework combining uncertainty quantification and decision theory that can be used to quantify the cost of cultural bias to a decision-maker.}
\end{figure}

The Columbia River Basin (CRB) provides a fruitful setting to quantify the cost of cultural bias in conservation decision-making (Box 1). Over the past few decades, the endangered, migratory Pacific salmon has faced increased predation pressure by California sea lions traveling into the Columbia River \parencite{wargo2019changes}. This conflict between two federally protected and broadly beloved species has strained decision-making, ultimately precipitating an amendment to the Marine Mammal Protection Act to allow legal euthanasia of these sea lions. Although the Columbia River salmon fisheries are among the most intensely managed and heavily studied fisheries in the world, one relatively understudied fish species – the Pacific lamprey – may play a role in restoring balance in this system (Figure 1).

The Pacific lamprey is an ancient, jawless fish that nearly all extant species in the CRB have co-evolved alongside \parencite{close2002ecological}. Importantly, the Pacific lamprey holds high cultural and spiritual value for CRB Indigenous peoples, and as a high value prey item for marine mammals, the lamprey may relieve predation pressure on salmon \parencite{CRITFC_2011}. Despite this value, however, Pacific lamprey has been driven to near extinction over the last century, largely due to cultural bias and exclusion by Euro-American society and management (Box 1). While modern management is reconciling with this history and increasing investment in lamprey restoration \parencite{clemens2017conservation}, the legacy of nearly a century of neglect is reflected in the data. Most data about salmon-lamprey-sea lion interactions come from observation systems designed for salmon, not lamprey \parencite{kostow2002oregon}, and understanding global, density-dependent dynamics is challenging, since observations are limited to a parametric space where lamprey are at much lower abundance than historic levels (Figure 2; Figure S1).


\begin{figure}[H]
    \centering    \includegraphics[width=1\textwidth]{Figure2_uncertainty_sources-01.png}
    \caption{Two sources of parametric uncertainty considered in the decision problem: \textbf{A.} Multi-species functional response uncertainty, derived from 1) uneven observation error of the true abundance of fish passing through the dam ($S^P$ and $L^P$ for salmon and lamprey, respectively) and 2) observation of local properties of the density-dependent multi-species functional response (i.e., observation space limited to low Pacific lamprey estuarine abundance, $E_L$). \textbf{B.} Parasitism uncertainty, or an unknown rate at which salmon ocean survival (proportion of juvenile salmon returning the the estuary, $E_S / J_S$) declines with juvenile lamprey abundance ($J_L$). The parameter $D_S$ describes this rate, and values are selected to cover the universe of possibilities, from a commensal lamprey-salmon marine relationship ($D_S$ = 9e7), to a scenario where salmon ocean survival declines to 0.04 at high lamprey densities ($D_S$ = 3e6).}
\end{figure}

Like all ecological observations, data reflecting salmon-lamprey-sea lion interactions are an opportunistic window into a complex generating statistical uncertainty in ecological inference. Yet these windows of opportunity are not neutral and have been shaped by human systems, and therefore statistical uncertainty can be artifacts of human bias (Figure 1). Translating the cost of this bias for decision-makers requires moving beyond abstract statistical scores and towards measurements on real-world objectives \parencite{boettiger2022forecast}. While the ecology and conservation literature provides little guidance on how to approach such a problem, we can draw from decision-theoretic methods to measure the cost of cultural bias in units relevant for a decision maker. 

In this study, we quantify the extent to which historical cultural bias against the Pacific lamprey impedes decision-making in the CRB. We first quantify uncertainty in the multi-species functional response (MSFR), or the functional form describing the rate of California sea lion predation as a function of the abundance of Pacific lamprey and spring Chinook salmon the Columbia River estuary \parencite{rosenbaum2024towards}. This Bayesian model quantifies global uncertainty in the MSFR and is developed to consider how systems of knowledge production have generated asymmetries in the observation-generating processes between fish species \parencite{carlen2024framework} (Figure 2). In the context of decision theory, uncertainty matters when it affects the choice of management action \parencite{mccarthy2007active}. The importance of uncertainty can be quantified as the Value of Information (VoI), or the degree to which unresolved uncertainty affects performance on management objectives, measured in the utility of the decision maker \parencite{runge2011uncertainty}. We then embed the quantified MSFR uncertainty into a decision model to quantify the value of resolving uncertainty related to the lamprey’s role in modulating salmon survival as both a parasite and predation offset for salmon (Figures 1-2). This value of resolving uncertainty - Value of Information - can be considered in the inverse, as the cost of a knowledge gap generated as an outcome of cultural bias. Through illustration with this Pacific lamprey case study, we provide a demonstrate a rigorous, quantitative approach to understanding the cost of cultural bias in conservation decision-making, moving from legacies of neglect reflected in data to performance on decision objectives.



\begin{center}
    \includegraphics[width=1\textwidth]{textbox_map.png}
\end{center}

\section{Results}

\subsection{Multi-species dynamical system}

We first built a stylized model of a Chinook salmon and Pacific lamprey dynamical system throughout their lifecycles as anadromous fish, including estuarine predation by California sea lions (CSL), freshwater population growth, and marine mortality (Box 1). We modeled Chinook salmon and Pacific lamprey interactions in 1) the marine environment through a density-dependent host-parasite relationship and 2) the estuarine environment through a density-dependent, multi-species functional response (MSFR), where the rate of California sea lion consumption is a function of the abundance of lamprey and salmon in the prey community. These two lamprey and salmon interactions were used represent the rate at which lamprey modulate salmon survival as both a parasite of salmon and a predation offset for salmon. The parameter values used to describe freshwater population growth and marine mortality were guided by historical observations and are described in Table S1, whereas the MSFR parameters were directly estimated with data. 

\subsection{Quantified multi-species functional response uncertainty}

To quantify uncertainty in the California sea lion (CSL) multi-species functional response (MSFR) (Equations 1-2), we developed a Bayesian model that relates CSL prey consumption data from gastro-intestinal diet analysis to the observed count of fish passing through Bonneville dam. The posterior distribution was used to quantify parametric uncertainty in the species-specific, density-dependent attack rates, $b_i$, species-specific handling times, $h_i$, and MSFR exponent, $q$ in Equations 1-2 (Table S2, Figure S2).

The Bayesian model was used to estimate abundance of prey available to the predator below Bonneville dam by accounting for error in observing the abundance of fish passing through the dam and imperfect dam passage efficiency (Figure 3A, Figure S3, Table S2). Across most sampled time points, the abundance of Pacific lamprey was less than 10\% of the abundance of Chinook salmon (Figure 3A). Yet despite this low abundance, Pacific lamprey were frequently over-represented in the CSL's diet (Figure 3A).

The relatively low abundance of lamprey in the prey community limited observation of global properties of the multi-species functional response, creating high parametric uncertainty in model regions with high lamprey density (Figure 3B-C). This observation of local properties of the density-dependent MSFR prevented distinction between a global preference for lamprey, where lamprey are favored at all densities, and negative switching, where lamprey switch from being over-represented to under-represented in the diet of CSLs as lamprey density increases (Figure 3B-C). The model's posterior samples captured this global uncertainty (Figure 3C; Table S2; Figure S2).

\begin{figure}[H]
    \centering    \includegraphics[width=0.74\textwidth]{Figure3_MSFR-01.png}
    \caption{Multi-species functional response, measuring the rate at which California sea lions (CSL) consume prey as a function of the composition of the prey community. \textbf{A.} Relationship between 1) the proportion of each fish species in the predator diet, measured with gastro-intestinal analysis of euthanized California sea lions, and 2) proportion of each fish species in the prey community, based on the mean posterior estimate of available prey, $L_A$ and $S_A$ for salmon and lamprey, respectively. The credibility intervals for the estimates of available prey, $L_A$ and $S_A$, are shown in Figure S3. \textbf{B.} Conceptual multi-species functional responses demonstrating how two multi-species functional response curves can have similar local properties in the observed parametric region shown in panel A, yet different global properties due to a large, unobserved parametric region. \textbf{C.} Estimated multi-species functional response, where each black line represents a posterior sample describing the predicted relationship between the proportion of each species in the CSL diet and the proportion of each species in the prey community. Four posterior samples are highlighted and are used in Figure 4. The dashed line indicates the 1:1 line, where CSL consume prey in equal proportion to their abundance in the prey community.}
\end{figure}

\subsection{Decision model}

We built a decision model to quantify the cost of lamprey-related uncertainty to a decision maker. The Pacific lamprey has both inherent value and spiritual and cultural value to humans, including as a First Food for Columbia River Basin tribes \parencite{close2002ecological, CRITFC_2011}. However, for the purposes of this decision problem, we primarily considered the value of Pacific lamprey through its ecological role as a predation offset for salmonids. We therefore framed the decision problem as finding the action, $a*$, that maximizes utility, $U$, where utility is defined as the equilibrium abundance of Chinook salmon returning to spawn in freshwater, $\hat{R}_S$.

In the decision problem, we considered a set of actions, $a \in A$, that represent changes in lamprey production, $\alpha_L$. Since the production parameter, $\alpha_L$, can be understood as the product of adult freshwater survival and intrinsic rate of population growth, an action of increasing $\alpha_L$ could refer to increasing adult lamprey survival (e.g., improving dam passage) and/or increasing lamprey reproduction (e.g., artificial lamprey propagation). 

While many uncertainties exist in this system, we considered two sources of parametric uncertainty (Figure 2): 1) uncertainty in the density-dependent California sea lion functional response, $F$, quantified above and 2) uncertainty in the strength of parasitism, $P$, or the rate at which salmon ocean survival declines as a function of lamprey abundance. We calculated the utility, $U(a, f, p)$, of each action, $a$, using the dynamical system model (Equations 1-7) under all expressions of functional response and parasitism uncertainty, $F$ and $P$.  

The optimal action, $a*$, that maximized utility (i.e., equilibrium abundance of salmon returning to the Columbia River basin) varied across different expressions of functional response and parasitism uncertainty (Figure 4A; Figure S4). The utility associated with the optimal action, $\text{max}_aU(a,f,p)$, also varied across different expressions of functional response and parasitism uncertainty (Figure 4B; Figure S4). In general, expressions of MSFR uncertainty associated with a global preference for lamprey (Figure 3C, blue line) corresponded to higher maximized expected utility, $\text{max}_aE_P[U(a,f,p)]$ (Figure 4A, blue line). Whereas expressions of MSFR uncertainty associated with negative switching (Figure 3C, green line) corresponded to lower maximized expected utility (Figure 4A, green line; Figure S5). Additionally, the bet-hedging strategy, or the action that maximizes utility over all uncertainty ($\text{max}_aE_{F,P}[U(a,f,p)]$), tended to be higher than the action that maximized utility after functional response uncertainty had been resolved ($\text{max}_aE_{P}[U(a,f,p)]$) (Figure 4A; Figure S5).

\begin{figure}[H]
    \centering    \includegraphics[width=1\textwidth]{Figure4_evpi-01.png}
    \caption{Quantifying the value of reducing lamprey-related uncertainty. \textbf{A.} Relationship between the expected utility and action for each multi-species functional response (MSFR) posterior sample highlighted in Figure 3C. Asterisk indicates that the expected utility, $E_P[U(a, f, p)]$, is calculated as the expectation over parasitism uncertainty, $P$ (Figure 2). The actions are the lamprey production, $\alpha_L$, relative to salmon production, $\alpha_S$, and can be understood as actions that either increase freshwater lamprey adult survival or the intrinsic rate of lamprey population growth. The points indicate the action that maximizes expected utility (i.e., $\text{max}_aE_P[U(a,f,p)]$) for each expression of functional response uncertainty. The black line indicates the expected utility over both parasitism and functional response uncertainty (i.e., $E_{F, P}[U(a,f,p)]$), and the black point indicates the action that maximizes utility over all uncertainty, or the bet-hedging strategy (i.e., $\text{max}_aE_{F,P}[U(a,f,p)]$). \textbf{B.} Violin plot of the maximized expected utility of all posterior samples, calculated as the expectation across parasitism uncertainty (i.e., $\text{max}_aE_P[U(a,f,p)]$). Colored points indicate the maximized expected utility of the selected posterior samples from panel A. The black point indicates the maximized expected utility across both parasitism and functional response uncertainty, or the utility associated with the bet-hedging strategy, $\text{max}_aE_{F,P}[U(a,f,p)]$. The purple dashed line indicates the expected utility after all uncertainty has been resolved, $E_{F,P}[\text{max}_aU(a,f,p)]$. The expected value of functional response and parasitism information (i.e., expected value of perfect information, EVPI) is therefore the difference between $E_{F,P}[\text{max}_aU(a,f,p)]$ and $\text{max}_aE_{F,P}[U(a,f,p)]$.}
\end{figure}

\subsection{Cost of lamprey uncertainty to a decision-maker}

In the context of this decision problem, the value of lamprey information was quantified as the expected improvement on the management objective (maximize abundance of Chinook salmon returning to the Columbia River Basin) if functional response uncertainty, $F$, and parasitism uncertainty, $P$, were reduced \parencite{runge2011uncertainty}. The utility associated with the bet-hedging strategy  ($\text{max}_aE_{F,P}[U(a,f,p)]$) was 125,343 fish, whereas the expected utility after all uncertainty had been resolved was 132,511 fish (Figure 4B). The expected value of perfect information, EVPI, was therefore 7167 fish (Figure 4B). Framed another way, the expected increase in Chinook salmon returning to the Columbia River Basin was 7167 fish, if lamprey uncertainty - generated as an outcome of cultural bias - was not impeding the choice of action.


\section{Discussion}

Evaluating the outcomes of conservation actions can be limited by structural uncertainty that arises from cultural bias. An ecological system with multiple interacting species in the Columbia River Basin provides a useful setting for quantifying the extent to which dominant value systems can impede conservation decision-making. Here, management has historically focused on high-value salmon fisheries and recent predation pressure by sea lions, while overlooking the Pacific lamprey. This focus has led to asymmetric measurement error and observation of local, rather than global, properties of the system, reflecting historical bias against the Pacific lamprey by dominant society.

By combining uncertainty quantification with decision theory, we have illustrated a method to measure the cost of overlooking this species in decision-making. The Bayesian approach to uncertainty quantification offers a flexible framework for representing uneven data-generating processes, and the Value of Information analysis facilitates systematic exploration of this uncertainty in a decision-making context. By unifying these approaches, we quantify the extent to which uncertainty about Pacific lamprey impedes decision-making about Chinook salmon, which can be understood as the cost of cultural bias, measured in units relevant to a decision maker. 



\subsection{Reconciling unknown unknowns}

Quantifying the cost of cultural bias to a decision maker through a Value of Information analysis is only possible if a spark brings to light previously ignored sources of uncertainty. In the case of the Pacific lamprey, Native American tribes began championing the importance of the species in the early 1990s, recognizing the declining numbers of lamprey and reflecting on the role of dominant Euro-American societal values in contributing to their decline \parencite{clemens2017conservation, close2002ecological, close2004traditional}. There is, however, not a VOI calculus for uncertainty that is not articulated, reflecting the philosophical conundrum of understanding the cost of “unknown unknowns” in scenarios where the truth lies far outside of the articulated hypotheses \parencite{wintle2010allocating}.

However, structural uncertainty in ecological modeling is often not random, as science is not neutral to power and uneven knowledge can arise systematically from political-economic structures and processes \parencite{silver2022fish}. Moving from “unknown unknowns” to quantifying the cost of “known unknowns” can involve interrogating systems of knowledge production through long-established ideas from the fields of environmental justice and science and technology studies. Investigating how environment and social difference are intertwined can provide valuable insight into how uneven distribution of power shapes conservation investment and research \parencite{walker2012environmental}. Considering knowledge as situated and knowing is a social process can be crucial for reflecting upon the systems of scientific practice \parencite{haraway2013situated, bloor2004sociology}. For instance, reflection on the relationship between a perceived ecological role and necropolitics – who should live and who should die – can be useful for interrogating the relationship between human psychology and conservation policy \parencite{chao2021beetle}. Integrating these ways of thinking into statistical ecology and decision making will be crucial for understanding how legacies of human structures can be present or create absences in ecological model development.

\subsection{Implications for adaptive management and structured decision-making}

Conservation decision-making is always bias- and value-laden; in fact, decision theory encourages value-based thinking and posits that values should be the driving force for decision-making \parencite{keeney1996value}. Danger, however, comes from a lack of reflection on where values enter the decision-making process. Decision analysis traditionally assumes that facts and values are separable: values are used to formulate the decision context and objectives, and facts are used to determine the consequences of actions \parencite{gregory2012structured}. Here we have shown, however, that historical values of a dominant society affect the ability to predict the consequences of actions, generating a quantifiable impediment to decision-making about salmon and lamprey in the Columbia River Basin.

While the inseparability of facts and values may challenge the conceptual foundation of decision analysis and structured decision making, including reflexive and relational thinking in adaptive management cycles represent a productive path forward. Adaptive management (AM) involves acknowledging uncertainty and seeking to reduce it through the process of management; AM can include two learning cycles, a technical learning cycle nested within a larger cycle of learning about the decision structure itself \parencite{williams2016technical}. To reduce uncertainty in how systems of power shape knowledge production and impede decision-making, this outer loop of double-loop learning could involve processes by which a decision analyst extends a relational understanding of the world to modelling, a practice known as “Situated Modelling” \parencite{klein2024situated}. This practice includes reflecting on the broader context in which knowledge is generated, questioning where modeling assumptions come from and what is left out, and broadening perspectives and ways of knowing at the decision-making table. 

\subsection{Limitations of an utilitarian understanding of information value}

In this analysis, we chose to evaluate the cost of lamprey uncertainty with respect to one decision objective - maximize the abundance of Chinook salmon returning to the CRB - among many possible decision objectives. This deliberate choice to frame the value of lamprey through the lens of its ecosystem service for salmon was chosen to represent how a historic, dominant value system that prioritized Pacific salmon can impede decision-making, even if the value system remains unchanged. 

The cost of cultural bias in decision-making, however, would change with a different decision objective or a different understanding of cost. While we translated signatures of cultural bias reflected in ecological data to a measurement of cost in units relevant to a decision maker (i.e., expected number of salmon), this utilitarian representation reflects a narrow understanding of information value. Cultural bias-derived uncertainty can impede decision-making in other ways, including generating conflict and diminishing trust among decision makers, thereby inhibiting the adaptive governance of the social-ecological system \parencite{folke2005adaptive}. 


\section{Materials and Methods}

We first describe a stylized representation of a Chinook salmon and Pacific lamprey dynamical system (Box 1). We then quantify uncertainty in estuarine predation using a Bayesian multi-species functional response (MSFR) model using sea lion gastro-intestinal diet data and fish passage data at Bonneville dam. Finally, we build a decision model that accounts for multiple sources of uncertainty in the dynamical system to evaluate management actions with respect to an objective of maximizing Chinook salmon equilibrium abundance. Using a Value of Information (VOI) analysis, we quantify how parametric uncertainty - generated as an outcome of cultural bias - impedes decision-making.

\subsection{Multi-species dynamical system}

Chinook salmon and Pacific lamprey are anadromous fish, migrating from freshwater rivers to the ocean and back to spawn in or near their natal streams \parencite{hess2023return}. While their lifecycles in the freshwater and marine systems extend across many years, we simplify the cycle to occur across one time step from $t$ to $t + 1$.

\subsubsection{Estuary predation}

The expected number of Pacific lamprey and Chinook salmon consumed by California sea lions (CSL) is described using a multi-species functional response (MSFR) that allows the attack rate to be dependent on prey density \parencite{rosenbaum2024towards}. Here, $\widetilde{F}_i$ represents the expected number of prey $i$ consumed per predator integrated across one day, $a_i$ represents the density-dependent attack rate of species $i$, $h_k$ represents the handling time for species $k = 1 \dots m$, and $E_i$ represents the abundance of species $i$ entering the estuary. 

\begin{equation}
\widetilde{F}_i = \frac{a_iE_i}{1+\sum_{k=1}^{m}a_kh_kE_k}    
\end{equation}

The density-dependent attack rate, $a_i$, is a function of coefficient, $b_i$, and exponent, $q$. When $q = 0$, the MSFR follows a Holling type 2 functional response, and when $q < 0$, the MSFR follows a Holling type 3 functional response (Figure S6). 

\begin{equation}
a_i = b_iE_i^q    
\end{equation}

The relationship between the abundance of fish returning to freshwater to spawn, $R_i$, and adults entering the estuary, $E_i$, of species $i \in \{L, S\}$ is therefore a function of the expected number of prey, $\widetilde{F}_i$, and the expected number of CSL days (i.e., daily abundance in estuary $\times$ days), $P$:

\begin{equation}
R_i = E_i - \widetilde{F}_i \times P  
\end{equation}

\subsubsection{Freshwater population growth}

The relationship between 1) the abundance of fish returning to the Columbia River Basin to spawn, $R_i$ and 2) the abundance of juveniles/smolts exiting Bonneville dam downstream, $J_i$, is described with the Ricker stock-recruit relationship, where $\alpha$ is a production parameter, and $K$ is the smolt or juvenile carrying capacity. The production parameter, $\alpha$, combines both freshwater mortality and the population intrinsic rate of growth.

We use the following growth equations to describe the relationship between the abundance of returning fish, $R_i$, and abundance of juveniles, $J_i$, for fish species $i$:
\begin{equation}
J_i = R_i \times \frac{\alpha_i}{(1+\beta_iR_i)}
\end{equation}
\begin{equation}
\beta_i = \frac{\alpha_i}{K_i}
\end{equation}

\subsubsection{Marine mortality}

In the marine environment, Pacific lamprey and Chinook salmon exhibit a host-parasite relationship \parencite{clemens2017conservation}. Salmonid host abundance in the marine environment is a principal factor in predicting Pacific lamprey returns to the Columbia River \parencite{murauskas2013relationships}, and while Chinook salmon are common lamprey hosts and exhibit wounds from Pacific lamprey bites \parencite{weitkamp2015seasonal, shevlyakov2010traumatization}, the effect of these bites on their physiological condition are not well understood \parencite{pelenev2008predator}.

The relationship between the abundance of lamprey juveniles entering the ocean, $J_L$, and the abundance of lamprey adults returning to the estuary, $E_L$, is described by the following equation, where survival increases as a function of salmon density, $J_S$, at a rate described by $D_L$:

\begin{equation}
E_L = J_L \times s^o \times (1 - \text{e}^{-D_L \times J_S})
\end{equation}

The relationship between the abundance of salmon smolts entering the ocean, $J_S$, and the abundance of salmon adults returning to the estuary, $E_S$, is described by the following equation, where survival decreases as a function of lamprey density, $J_L$, at a rate scaled by $D_S$:

\begin{equation}
E_S = J_S \times s^o \times (1 - \text{e}^{-D_S/J_L})
\end{equation}

Here, density-independent ocean survival, $s^o$, is shared between both species.

\subsection{Quantifying uncertainty in the multi-species functional response (MSFR)}

Below we outline the Bayesian model formulation used to estimate the multi-species functional response (MSFR) parameters with California sea lion gastro-intestinal diet data and fish count data at Bonneville dam (Tables S3-S6). More information on data sources and model fitting can be found in Supporting Information.

\subsubsection{Chinook salmon prey abundance}

The observed number of Chinook salmon passing through the Bonneville dam fish ladders in time window, $t$, $S^P_{t}$, is drawn from a Poisson distribution with $\widetilde{S^P_{t}}$ representing the expected number of salmon passed (Table S4):

\begin{equation}
S^P_{t} \sim \text{Poisson}(\widetilde{S^P_{t}})
\end{equation}

The number of salmon available for CSL consumption, $S^A_t$ is greater than the number of salmon passing through the dam, as the dam passage efficiency is less than one. The expected number of salmon passed, $\widetilde{S^P_t}$, is therefore drawn from a Binomial distribution with the size as the difference between the number of salmon available for consumption, $S^A_t$, and the total number of salmon consumed across all euthanized CSL individuals at time $t$, $\sum^{O_t}_{j=1}{F^S_{t, j}}$. Here, $O_t$ is the total number of individuals euthanized at time $t$, and $F^S_{t,j}$ is the number of Chinook salmon enumerated in the digestive tract of CSL individual, $j$. The probability is the salmon passage efficiency at Bonneville dam, $p_S$, \parencite{frick2008adult}:

\begin{equation}
\widetilde{S^P_t} \sim \text{Binomial}(S^A_t - \sum^{O_t}_{j=1}{F^S_{t, j}},p_S)
\end{equation}

\subsubsection{Pacific lamprey prey abundance}

In the years 2017 and 2019 when the lamprey night-time and LPS counts were recorded, the observed number of lamprey passing through all passage structures (LPS + fish ladders) in time window $t$, $L^P_t$, is drawn from a Poisson distribution with $\widetilde{L^P_t}$ representing the expected number of lamprey passed (Table S4):

\begin{equation}
L^P_t \sim \text{Poisson}(\widetilde{L^P_t})\qquad \text{for } t \in \lambda^E
\end{equation} 
where $\lambda^E$ represents the set of time windows associated with CSL euthanasia in 2017 or 2019.

In the years when the lamprey night-time and LPS counts were not recorded, the observed number of lamprey passing through all passage structures, $L^P_t$, is drawn from a Binomial distribution, with the size parameter as the expected number of lamprey passed, $\widetilde{L^P_t}$, and probability of detection in daytime visual counts, $p^D_t$ (Table S4):

\begin{equation}
L^P_t \sim \text{Binomial}(\widetilde{L^P_t}, p^D_t) \qquad \text{for } t \in \lambda^I
\end{equation} 
where $\lambda^I$ represents the set of time windows associated with CSL euthanasia in 2018 or 2021-2023.

The probability of detection in daytime visual counts during the time window $t$, $p^D_t$, is drawn from a Beta distribution:

\begin{equation}
p^D_t \sim \text{Beta}(\alpha^D, \beta^D) \qquad \text{for } t \in \lambda^I
\end{equation}

These beta distribution hyperparameters were informed by the passed lamprey recorded via visual count, $C_d^V$, and the total recorded passed lamprey across all passage structures and all times of day, $C_d^T$, for each monitored three-day interval, $d$, in 2017 and 2019 (Table S5, Figure S7). $C_d^V / C_d^T$ therefore represents the fraction of all passed lamprey that were visually counted during the day:

\begin{equation}
C_d^V / C_d^T \sim \text{Beta}(\alpha^D, \beta^D)
\end{equation}

The number of lamprey available for CSL consumption, $L^A_t$ is greater than the number of lamprey passing through the dam, as the dam passage efficiency is less than one. The expected number of lamprey passed, $\widetilde{L^P_t}$, is drawn from a Binomial distribution with the size as the difference between the number of lamprey available for consumption, $L^A_t$, and the number of lamprey consumed across all euthanized CSL individuals at time $t$, $\sum^{O_t}_{j=1}{F^L_{t, j}}$. Here, $F^L_{t,j}$ is the number of lamprey enumerated in the digestive tract of CSL individual, $j$. The probability is the lamprey passage efficiency at Bonneville dam, $p_L$, \parencite{moser2002passage}:

\begin{equation}
\widetilde{L^P_t} \sim \text{Binomial}(L^A_t - \sum^{O_t}_{j=1}{F^L_{t, j}},p_L)
\end{equation}

\subsubsection{Consumed prey}
The number of prey consumed by each CSL individual is likely to deviate from the expected number of prey consumed due to heterogeneity in preference and behavior across CSL individuals. Model process error is therefore incorporated as Poisson-distributed variation in CSL consumption rates. The observed number of Chinook salmon and Pacific lamprey consumed by CSL individual $j$, $F^S_{t,j}$ and $F^L_{t,j}$, respectively, is drawn from a Poisson distribution with the expected number of prey consumed, $\widetilde{F^S_{t}}$ and $\widetilde{F^L_{t}}$, as the rate parameters, respectively (Table S6):

\begin{equation}
F^S_{t, j} \sim \text{Poisson}(\widetilde{F^S_{t}})
\end{equation}
\begin{equation}
F^L_{t, j} \sim \text{Poisson}(\widetilde{F^L_{t}})
\end{equation}

\subsubsection{Multi-species functional response (MSFR)}

The expected number of prey consumed, $\widetilde{F_{i,t}}$, of species $i$ is related to the available number of prey, $A_{i,t} = \{S^A_{t}, L^A_{t}\}$, using the MSFR described in Equations 1-2.

\begin{equation}
\widetilde{F_{i,t}} = \frac{a_{i,t}A_{i,t}}{1+\sum_{k=1}^{m}a_{k,t}h_kA_{k,t}}    
\end{equation}  

\begin{equation}
a_{i,t} = b_iA_{i,t}^q    
\end{equation}



\subsection{Decision model}

We then developed a decision model, where we framed the decision problem as finding the action, $a*$, that maximizes utility, $U$, where utility is defined as the equilibrium abundance of Chinook salmon returning to spawn in freshwater, $\hat{R}_S$. The set of actions, $a \in A$, under consideration in the decision problem include changes in lamprey production, $\alpha_L$. While many uncertainties exist in this system, we considered two sources of parametric uncertainty (Figure 2): 1) functional response uncertainty, $F$ (Figure 3) and 2) parasitism uncertainty, $P$, describing uncertainty in the rate at which salmon ocean survival declines as a function of lamprey abundance.

We calculated the utility of each action, $a$, using the dynamical system model (Equations 1-7) under all combinations of functional response and parasitism uncertainty, $F$ and $P$. Here, $U(a, f, p)$ is the utility associated with taking action $a$ assuming $f \in F$ and $p \in P$. The set $F$ corresponds to 1000 randomly selected samples from the MSFR model posterior distribution, and the set $P$ corresponds to three values of $D_S$ (Equation 7) with equal prior probability (Figure 2, Table S1). The utility, or the equilibrium salmon return abundance ($\hat{R}_S$), was calculated through simulation, starting at an arbitrary abundance of salmon and lamprey returns, $R_S$ and $R_L$, and finding the mean salmon return abundance, $\hat{R}_S$, after removing transient dynamics.

\subsubsection{Value of Information}

We then used a Value of Information analysis to calculate the value of lamprey information. In classical decision theory, the expected value of perfect information (EVPI), is the difference between the expected value of an optimal action after new information has been collected and the expected value of an optimal action before new information has been collected:

\begin{equation}
\text{EVPI} = E_{F,P}[\text{max}_aU(a,f,p)] - \text{max}_aE_{F,P}[U(a,f,p)]
\end{equation} 

The first term in Equation 19 represents the expected utility once all uncertainty has been resolved, because the optimal action is chosen after perfectly knowing the multi-species functional response relationship and how parasitism strength scales with lamprey density. The second term in Equation 19 is the bet-hedging strategy, or the expected value associated with the action that maximizes utility over all sources of uncertainty. The difference between these two terms, EVPI, is therefore defined as the expected increase in utility if lamprey uncertainty is resolved.


\printbibliography[]

\end{document}
